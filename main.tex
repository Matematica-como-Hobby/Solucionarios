\documentclass[portuguese,12pt,a4paper]{book}
\usepackage[latin1]{inputenc}
\usepackage[T1]{fontenc}
\usepackage{graphicx}
\usepackage{mathtools}
\usepackage{amssymb}
\usepackage{amsthm}
\usepackage{babel}
\usepackage[colorlinks=true]{hyperref}
%\usepackage{biblatex}
%\addbibresource{bibliografia.bib}

\title{A teoria dos conjuntos e os fundamentos da matem�tica - Solu��es}

\newcommand{\cqd}{\hfill $\square$}
\newenvironment{solucao}[1][]{\noindent\textbf{Solu��o:} }{\cqd}
\newtheorem{exercicio}{Exerc�cio}

\begin{document}
	
	\maketitle
	\tableofcontents
	
	\chapter{Hist�ria e Motiva��o}

\begin{exercicio}
	Exiba uma bije��o entre o conjunto dos n�meros inteiros e os naturais.
\end{exercicio}
\begin{solucao}
	Defina a fun��o $f:\mathbb{N}\to\mathbb{Z}$ por $f(n)=-\frac{n}{2}$ se $n$ � par e por $f(n)=\frac{n+1}{2}$ se $n$ � �mpar. $f$ � uma bije��o.
\end{solucao}

\begin{exercicio}
	Prove que qualquer subconjunto infinito dos n�meros naturais � enumer�vel.
\end{exercicio}
\begin{solucao}
	Lembre: dizemos que um conjunto $X$ � \textit{enumer�vel} se � finito ou � equipotente a $\mathbb{N}$, isto �, se existe uma bije��o $f:\mathbb{N}\to X$.
	
	Todo conjunto finito � enumer�vel. Est�o vamos olhar apenas o caso $X$ ser infinito. Defina $f:\mathbb{N}\to X$ da seguinte maneira: $f(0)$ � o menor elemento de $X$; $f(1)$ � o segundo menor elemento de $X$; $f(2)$ � o terceiro menor elemento de $X$, e assim por diante. De maneira mais rigorosa, $f$ � definida indutivamente por: $f(0)=\min X$; uma vez escolhidos, $f(0),f(1),\ldots,f(n)$, definimos $f(n+1)=\min(X\backslash\{f(0),f(1),\ldots,f(n)\})$. Observe que $f$ � crescente, logo injetiva. Tamb�m � sobrejetiva, pois do contr�rio, se existe um $x\in X\backslash f(\mathbb{N})$, ent�o $x\in X\backslash\{f(0),f(1),\ldots,f(n)\}$ para cada $n\in\mathbb{N}$. Da maneira como constru�mos $f$ conclu�mos que $x>f(n)\ \forall n\in\mathbb{N}$ e assim $f(\mathbb{N})$ � um conjunto limitado, logo finito. Como $f:\mathbb{N}\to f(\mathbb{N})$ � uma bije��o, $\mathbb{N}$ seria finito, um absurdo. Portanto $f:\mathbb{N}\to X$ � uma bije��o, logo $X$ � enumer�vel.
\end{solucao}

\begin{exercicio}
	Na bije��o que constru�mos entre os n�meros naturais e os polin�mios, encontre o polin�mio associado ao n�mero 30.
\end{exercicio}
\begin{solucao}
	\begin{equation*}
		\begin{array}{|ccc|ccc|ccc|}
			\hline
			0  & \mapsto & -x-1  & 11 & \mapsto & -x-2       & 21 & \mapsto & -2x^2-2x-1 \\ \hline
			1  & \mapsto & -x    & 12 & \mapsto & -x+2       & 22 & \mapsto & -2x^2-2x   \\ \hline
			2  & \mapsto & -x+1  & 13 & \mapsto & x-2        & 23 & \mapsto & -2x^2-2x+1 \\ \hline
			3  & \mapsto & x-1   & 14 & \mapsto & x+2        & 24 & \mapsto & -2x^2-2x+2 \\ \hline
			4  & \mapsto & x     & 15 & \mapsto & 2x-2       & 25 & \mapsto & -2x^2-x-2  \\ \hline
			5  & \mapsto & x+1   & 16 & \mapsto & 2x-1       & 26 & \mapsto & -2x^2-x-1  \\ \hline
			6  & \mapsto & -2x-2 & 17 & \mapsto & 2x         & 27 & \mapsto & -2x^2-x    \\ \hline
			7  & \mapsto & -2x-1 & 18 & \mapsto & 2x+1       & 28 & \mapsto & -2x^2-x+1  \\ \hline
			8  & \mapsto & -2x   & 19 & \mapsto & 2x+2       & 29 & \mapsto & -2x^2-x+2  \\ \hline
			9  & \mapsto & -2x+1 & 20 & \mapsto & -2x^2-2x-2 & 30 & \mapsto & -2x^2-2    \\ \hline
			10 & \mapsto & -2x+2 &    &         &            &    &         &            \\ \hline
		\end{array}
	\end{equation*}
	Portanto o polin�mio associado ao n�mero 30 � $-2x^2-2$.
\end{solucao}

\begin{exercicio}
	Na bije��o que constru�mos entre os n�meros naturais e os n�meros alg�bricos, encontre o n�mero natural associado ao n�mero $\sqrt{3}$
\end{exercicio}
\begin{solucao}
	content
\end{solucao}

\begin{exercicio}
	Suponha que, em um conjunto infinito, existe uma forma de representar cada elemento do conjunto com uma sequ�ncia finita de s�mbolos, dentre um conjunto finito de s�mbolos. Mostre que esse conjunto � enumer�vel e use esse resultado diretamente para mostrar que os conjuntos dos n�meros racionais e dos n�meros alg�bricos s�o enumer�veis.
\end{exercicio}
\begin{solucao}
	Seja $\Sigma=\{\sigma_1,\ldots,\sigma_m\}$ o conjunto de s�mbolos. Para cada $n\in\mathbb{N}$, seja $S_n=\{(\tau_1,\ldots,\tau_n):\tau_1,\ldots,\tau_n\in \Sigma\}$ e ponha $S=\bigcup_{n=1}^\infty S_n$. $S$ � enumer�vel por ser uma reuni�o enumer�vel de conjuntos enumer�veis (finitos).\footnote{\cite[p. 51]{Lima16}} Seja $X$ o conjunto infinito do exerc�cio. A suposi��o de que cada elemento de $X$ possa ser representado como uma sequ�ncia finita de s�mbolos significa que existe uma fun��o $f:X\to S$. Esta fun��o deve ser injetiva, pois n�o podemos usar a mesma sequ�ncia de s�mbolos para representar dois ou mais elementos distintos em $X$. Ent�o $f: X\to f(X)\subset S$ � bijetiva e como $f(X)$ � enumer�vel, por ser um subconjunto de um conjunto enumer�vel conforme o Exerc�cio 2, $X$ � enumer�vel.
	
	Todo n�mero racional pode ser escrito como $a/b$, com $a,b\in \mathbb{Z}$, $b\neq 0$. Ambos $a$ e $b$ s�o escritos como uma sequ�ncia finita de s�mbolos (os algarismos de 0 a 9 e o sinal $+$ ou $-$). Pelo resultado demonstrado acima, $\mathbb{Q}$ � enumer�vel.
	
	Cada polin�mio com coeficientes inteiros pode ser escrito como uma sequ�ncia finita de s�mbolos, logo o conjunto formado por esses polin�mios � enumer�vel. O conjunto dos n�meros alg�bricos � a uni�o dos conjuntos das ra�zes de cada um desses polin�mios. Como essa � uma uni�o enumer�vel (o conjunto desses polin�mios � enumer�vel) de conjuntos finitos (cada polin�mio possui um n�mero finito de ra�zes), o conjunto dos n�meros alg�bricos � enumer�vel.
\end{solucao}

\begin{exercicio}
	Imagine que o hotel de Hilbert, com uma quantidade infinita enumer�vel de quartos, todos ocupados, receba infinitos trens com infinitos vag�es e cada vag�o com infinitos passageiros (todas essas quantidades enumer�veis). Como o gerente pode alocar todos os atuais h�spedes em quartos separados?
\end{exercicio}
\begin{solucao}
	Cada passageiro tem uma identifica��o $(p,q,r)$: $p$ = n�mero do trem; $q$ = n�mero do vag�o; $r$ = n�mero do assento. Cada h�spede no quarto $n$ ser� realocado para o quarto $2n-1$, e cada passageiro com ID $(p,q,r)$ ser� hospedado no quarto $2^p3^q(2r-1)$.
	
	Isto pode ser generalizado: h�spede $n\mapsto$ quarto $2n-1$; passageiro $(a_1,\ldots,a_m)\mapsto$ quarto $2^{a_1}3^{a_2}\cdots p_{m-1}^{a_{m-1}}(2a_m-1)$, em que $2,3,\ldots,p_{m-1}$ s�o n�meros primos.
\end{solucao}

\begin{exercicio}
	Imagine, agora, um hotem maior ainda, com um quarto para cada n�mero real, totalmente ocupado. Um �nibus igualmente gigantesco, com um passageiro para cada n�mero real, chega ao hotel. Como o gerente pode fazer para rearranjar os h�spedes para acolher os novos visitantes, sempre em quartos separados?
\end{exercicio}
\begin{solucao}
	\begin{itemize}
		\item H�spede $x\mapsto$ quarto $\arctan x$.
		\item Passageiro $y\mapsto$ quarto $y+\pi/2$ se $y\geq0$ ou $y-\pi/2$ se $y<0$. 
	\end{itemize}
	Note que o hotel ficar� com um quarto vago (o de n�mero $-\pi/2$).
\end{solucao}
	\chapter{A linguagem da Teoria dos Conjuntos}

\setcounter{ex}{0}

\begin{exercicio}
	Usando a linguagem de primeira ordem da teoria de conjuntos, escreva fórmulas para representar as seguintes frases.
\end{exercicio}

\begin{solucao}
\begin{enumerate}
	\item[a)] Não existe o conjunto de todos os conjuntos.
    \item[] $\neg \exists x \forall y (y \in x) \equiv \forall x \exists y (y \notin x)$
	\item[b)] Existe um único conjunto vazio.
    \item[] $\exists ! x \forall y (y \notin x)$
	\item[c)] x é um conjunto unitário
    \item[] $\exists ! y (y \in x)$
	\item[d)] Existe um conjunto que tem como elemento apenas o conjunto vazio
    \item[] $\exists x \forall y ((y \in x) \leftrightarrow y = \phi)$
	\item[e)] y é o conjunto dos subconjuntos de x 
    \item[] $\forall z (z \in y \leftrightarrow \forall w (w \in z \to w \in x))$.
    
\end{enumerate}

\end{solucao}

\begin{exercicio}
	Marque as ocorrências de variáveis livres nas fórmulas abaixo
\end{exercicio}

\begin{enumerate}
	\item[a)] $(\forall x (x=y)) \rightarrow (x \in y ) $
    \item[] $x$ e $y$
	\item[b)] $ \forall x ((x=y) \rightarrow (x \in y))$
    \item[] $y$
	\item[c)] $\forall x(x=x) \rightarrow (\forall y \exists Z ((x=y) \land (y=z)) \rightarrow \neg(x\in y))$
    \item[] $x$
	\item[d)] $ \forall x \exists y(\neg(x=y) \land \forall z ((x \in y) \leftrightarrow \forall w ((w \in z ) \rightarrow (w \in x )))) $
    \item[] Não há variáveis livres.
	\item[e)] $(x=y)\rightarrow \exists (x=y) $
    \item[] $x$ e $y$
\end{enumerate}

\begin{exercicio}
	Escreva as subfórmulas de cada fórmula do exercício 2.
\end{exercicio}

\begin{solucao}

\begin{enumerate}
	\item 
	\begin{enumerate}
		\item $(\forall x (x = y)) \rightarrow (x \in y)$
		\item $(\forall x (x = y))$
		\item $(x = y)$
		\item $(x \in y)$
	\end{enumerate}
	
	\item 
	\begin{enumerate}
		\item $\forall x ((x = y) \rightarrow (x \in y))$
		\item $(x = y) \rightarrow (x \in y)$
		\item $(x = y)$
		\item $(x \in y)$
	\end{enumerate}
	
	\item 
	\begin{enumerate}
		\item $\forall x (x = x) \rightarrow (\forall y \exists z (((x = y) \land (y = z)) \rightarrow \neg (x \in y)))$
		\item $\forall x (x = x)$
		\item $(x = y)$
		\item $\forall z \exists y (((x = y) \land (y = z)) \rightarrow \neg (x \in y))$
		\item $((x = y) \land (y = z))$
		\item $(x = y)$
		\item $(y = z)$
		\item $\neg (x \in y)$
		\item $(x \in y)$
	\end{enumerate}
	
	\item 
	\begin{enumerate}
		\item $(x = y) \rightarrow \exists y (x = y)$
		\item $(x = y)$
		\item $\exists y (x = y)$
		\item $(x = y)$
	\end{enumerate}
\end{enumerate}

\end{solucao}
	\chapter{Primeiros Axiomas}

\setcounter{ex}{0}

\begin{exercicio}
	Usando o axioma da extens?o, prove que $\{\emptyset\}$ e $\{\{\emptyset\}\}$ s?o conjuntos diferentes.
\end{exercicio}
\begin{solucao}
	content
\end{solucao}

\begin{exercicio}
	Para cada par de conjuntos abaixo, decida qual(is) dos s?mbolos $\in$ e $\subset$ torna(m) a f?rmula verdadeira (assumindo que esses conjuntos existem). Lembre-se de que a resposta tamb?m pode ser ambos os s?mbolos ou nenhum deles. Justifique cada resposta.
	\begin{enumerate}[label=(\alph{*})]
		\item $\{\emptyset\}\ldots\{\emptyset,\{\emptyset\}\}$
		\item $\{\emptyset\}\ldots\{\{\emptyset\}\}$
		\item $\{1,2,3\}\ldots\{\{1\},\{2\},\{3\}\}$
		\item $\{1,2,3\}\ldots\{\{1\},\{1,2\},\{1,2,3\}\}$
		\item $\{1,2\}\ldots\{1,\{1\},2,\{2\},\{3\}\}$
		\item $\{\{1\},\{2\}\}\ldots\{\{1,2\}\}$
	\end{enumerate}
\end{exercicio}
\begin{solucao}
	content
\end{solucao}

\begin{exercicio}
	Seja $x$ o conjunto $\{\emptyset,\{\emptyset\},\emptyset,\{\emptyset,\{\emptyset\}\}\}$.
	\begin{enumerate}[label=(\alph{*})]
		\item Quantos elementos tem o conjunto $x$?
		\item Descreva todos os subconjuntos de $x$.
		\item Descreva, usando chaves e v?rgula, o conjunto de todos os subconjuntos de $x$.
		\item Quantos elementos o conjunto dos subconjuntos de $x$ possui?
		\item Prove que o conjunto $x$ existe.
	\end{enumerate}
\end{exercicio}
\begin{solucao}
	content
\end{solucao}

\begin{exercicio}
	Prove que para todos conjuntos $x$ e $y$
	\begin{enumerate}[label=(\alph{*})]
		\item $x\subset x$;
		\item $x\in y$ se, e somente se, $\{x\}\subset y$;
		\item $\bigcup\mathcal{P}(x)=x$;
		\item se $x\subset y$, ent?o $\bigcup x\subset \bigcup y$.
	\end{enumerate}
\end{exercicio}
\begin{solucao}
	content
\end{solucao}

\begin{exercicio}
	Escreva uma f?rmula de primeira ordem, na linguagem da teoria dos conjuntos, com quatro vari?veis livres, que represente o conjunto $\{x,y,z\}$.
\end{exercicio}
\begin{solucao}
	content
\end{solucao}

\begin{exercicio}
	Escreva os seguintes conjuntos, listando seus elementos entre chaves:
	\begin{enumerate}[label=(\alph{*})]
		\item $\bigcup\{\{0,1\},\{\{1\}\},\{1,2\},\{\{1,2\}\}\}$;
		\item $\mathcal{P}(\{\emptyset,\{\emptyset\}\})$.
	\end{enumerate}
\end{exercicio}
\begin{solucao}
	content
\end{solucao}

\begin{exercicio}
	Prove que n?o existe o conjunto de todos os conjuntos unit?rios.

	\emph{Dica:} Assuma, por absurdo, a exist?ncia do conjunto de todos os conjuntos unit?rios e prove a exist?ncia do conjunto de todos os conjuntos.
\end{exercicio}
\begin{solucao}
	content
\end{solucao}

\begin{exercicio}
	Prove que para todo conjunto $X$ existe o conjunto
	$$\{\{x\}:x\in X\}$$
\end{exercicio}
\begin{solucao}
	content
\end{solucao}

\begin{exercicio}
	Senso $x$ um conjunto n?o vazio, prove que
	\begin{enumerate}[label=(\alph{*})]
		\item $\forall y(y\in x\to(\bigcap x\subset y))$;
		\item $x\subset y\to \bigcap y\subset \bigcap x$.
	\end{enumerate}
\end{exercicio}
\begin{solucao}
	content
\end{solucao}

\begin{exercicio}
	Escreva na linguagem da l?gica de primeira ordem, sem abreviaturas, a seguinte f?rmula:
	$$x\in\bigcup\bigcap(y\cup(w\backslash z)).$$
\end{exercicio}
\begin{solucao}
	content
\end{solucao}

\begin{exercicio}
	Usando o axioma da regularidade, prove que:
	\begin{enumerate}[label=(\alph{*})]
		\item n?o existem $x,y,z$ tais que $x\in y,y\in z$ e $z\in x$;
		\item n?o existem $w,x,y,z$ tais que $w\in x,x\in y,y\in z$ e $z\in w$.
	\end{enumerate}
\end{exercicio}
\begin{solucao}
	content
\end{solucao}

\begin{exercicio}
	Prove que n?o existe $x$ tal que $\mathcal{P}(x)=x$.
\end{exercicio}
\begin{solucao}
	content
\end{solucao}

\begin{exercicio}
	Escreva o conjunto $\mathcal{P}(3\backslash 1)$, utilizando apenas os seguintes s?mbolos: as chaves, a v?rgula e o s?mbolo de conjunto vazio.
\end{exercicio}
\begin{solucao}
	content
\end{solucao}

\begin{exercicio}
	Prove, a partir dos axiomas de Peano, os seguintes teoremas:
	\begin{enumerate}[label=(\alph{*})]
		\item Todo n?mero natural ? diferente do seu sucessor.
		\item Zero ? o ?nico n?mero natural que n?o ? sucessor de nenhum n?mero natural.
	\end{enumerate}
\end{exercicio}
\begin{solucao}
	content
\end{solucao}

\begin{exercicio}
	Prove que:
	\begin{enumerate}[label=(\alph{*})]
		\item para todo $n\in\omega$, $\emptyset\in n$ ou $\emptyset=n$;
		\item para todos $n,m\in\omega$, se $m\in n$, ent?o $m\subset n$.
	\end{enumerate}
\end{exercicio}
\begin{solucao}
	content
\end{solucao}

\begin{exercicio}
	A uni?o de dois conjuntos indutivos ? necessariamente um conjunto indutivo? Justifique sua resposta.
\end{exercicio}
\begin{solucao}
	content
\end{solucao}
	\chapter{Produto Cartesiano, Relações e Funções}

\setcounter{ex}{0}

\begin{exercicio}
	content
\end{exercicio}
\begin{solucao}
	content
\end{solucao}

\begin{exercicio}
	Prove que, se $A\subset C$ e $B\subset D$, então $A\times B\subset C\times D$.
\end{exercicio}
\begin{solucao}
	Se $A=\emptyset$ ou $B=\emptyset$, então pelo Exercício 1, $A\times B=\emptyset$ e temos trivialmente $A\times B\subset C\times D$. Suponha então que $A\times B\neq\emptyset$. Dado um par ordenado $(a,b)\in A\times B$, temos que $a\in A$ e $b\in B$. Uma vez que $A\subset C$ e $B\subset D$, então $a\in C$ e $b\in D$, logo $(a,b)\in C\times D$. Portanto $A\times B\subset C\times D$.
\end{solucao}

\begin{exercicio}
	content
\end{exercicio}
\begin{solucao}
	content
\end{solucao}

\begin{exercicio}
	Descreva todos os elementos de $\mathcal{P}(2\times 2)$.
\end{exercicio}
\begin{solucao}
	$$2=\{0,1\}\Rightarrow 2\times2=\{(0,0),(0,1),(1,0),(1,1)\}$$
	\begin{eqnarray*}
		\Rightarrow\mathcal{P}(2\times 2)&=&\{\emptyset,\{(0,0)\},\{(0,1)\},\{(1,0)\},\{(1,1)\},\{(0,0),(0,1)\}, \\
		&&\{(0,0),(1,0)\},\{(0,0),(1,1)\},\{(0,1),(1,0)\},\{(0,1),(1,1)\} \\
		&&\{(0,0),(0,1),(1,0)\},\{(0,0),(0,1),(1,1)\},\{(0,0),(1,0),(1,1)\} \\
		&&\{(0,1),(1,0),(1,1)\},\{(0,0),(1,0),(0,1),(1,1)\}\}
	\end{eqnarray*}
\end{solucao}

\begin{exercicio}
	content
\end{exercicio}
\begin{solucao}
	content
\end{solucao}

\begin{exercicio}
	Prove que $x^0=1$, para todo conjunto $x$, e explique o que isso significa.
\end{exercicio}
\begin{solucao}
	$x^0$ é o conjunto das funções de $0=\emptyset$ em $x$. Perceba inicialmente que $\emptyset\times x=\emptyset$ pelo Exercício 1, logo $\emptyset$ é uma relação. Afirmamos que essa relação é uma função. Com efeito, dizer que $\emptyset$ é função significa que se $a\emptyset b$ e $a\emptyset c$, então $b=c$. As afirmações $a\emptyset b$ e $a\emptyset c$ são ambas falsas, logo a sentença $((a\emptyset b)\wedge(a\emptyset c))\to(b=c)$ é verdadeira (independente de $b=c$ ser verdadeiro ou falso). Portanto $\emptyset$ é uma função. Note que $\dom(\emptyset)=\emptyset$.
	
	Afirmamos agora que $\emptyset$ é a única função cujo domínio é $\emptyset$. Com efeito, se $f:\emptyset\to x$ é uma função e $f\neq\emptyset$, então existe um par $(a,b)\in f$. Por definição de relação, $a\in\emptyset$ e $b\in x$, mas $a\in\emptyset$ é um absurdo. Logo $f=\emptyset$.
	
	Portanto $x^0=\{\emptyset\}=1$.
\end{solucao}

\begin{exercicio}
	content
\end{exercicio}
\begin{solucao}
	content
\end{solucao}

\begin{exercicio}
	Para qual(is) das seguintes definições alternativas de pares ordenados o teorema 4.2 continua valendo? Justifique.
	\begin{enumerate}[label=(\alph{*})]
		\item $(a,b)=\{a,\{b\}\}$;
		\item $(a,b)=\{\{a\},\{b\}\}$;
		\item $(a,b)=\{\{a\},\{\{b\}\}\}$;
		\item $(a,b)=\{\{0,a\},\{1,b\}\}$.
	\end{enumerate}
\end{exercicio}
\begin{solucao}
	\begin{enumerate}[label=(\alph{*})]
		\item Não vale, pois $(1,1)=(\{1\},0)=\{1,\{1\}\}$.
		\item Não vale, pois $(a,b)=\{\{a\},\{b\}\}=\{\{b\},\{a\}\}=(b,a)$, e isto contraria o teorema se $a\neq b$.
		\item Não vale, pois $(1,1)=(\{1\},0)=\{\{1\},\{\{1\}\}\}$.
		\item Suponha que $(a,b)=(c,d)$. Então $\{\{0,a\},\{1,b\}\}=\{\{0,c\},\{1,d\}\}$. Se $a=b$, então $(a,b)=\{\{0,a\},\{1,a\}\}$, logo $\{0,c\}=\{0,a\}$ (e $\{1,d\}=\{1,a\}$) ou $\{0,c\}=\{1,a\}$ (e $\{1,d\}=\{0,a\}$). No primeiro caso teremos $c=a$ e $d=a$, logo $d=b$. No segundo, não podemos ter $c=a$, pois assim teremos $0=1$. Então $c=1$ e $a=0$. Da mesma forma, $d=0$ e $a=1$. Disto concluímos que $a=0$ e $a=1$, um absurdo. Portanto este segundo caso não pode ocorrer. Assim só vale o primeiro caso, que é $c=a$ e $d=b(=a)$.
		
		Suponha agora que $a\neq b$. Então $\{0,a\}=\{0,c\}$ e $\{1,b\}=\{1,d\}$, logo $a=c$ e $b=d$, ou $\{0,a\}=\{1,d\}$ e $\{1,b\}=\{0,c\}$, mas este segundo caso não pode ocorrer pois isto gerará a conclusão de que $0=1$, como visto acima.
		
		Em ambos os casos esdudados, $a=c$ e $b=d$, logo o teorema é válido para esta definição de par ordenado.
	\end{enumerate}
\end{solucao}

\begin{exercicio}
	content
\end{exercicio}
\begin{solucao}
	content
\end{solucao}

\begin{exercicio}
	Dê exemplo de uma função sobrejetora de $\omega$ em $\omega$ que não é injetora.
\end{exercicio}
\begin{solucao}
	Pelo Teorema Fundamental da Aritmética, todo número natural $n\geq2$ pode ser escrito como um produto de números primos $n=p_1\cdots p_m$, e este produto é único a menos da ordem dos fatores. Esse teorema nos permite definir uma função $f:\omega\to\omega$ pondo $f(0)=0$, $f(1)=1$ e, para cada $n\geq2$, $f(n)=m$, em que $m$ é a quantidade de fatores primos na decomposição de $n$. $f$ é sobrejetiva, pois, por exemplo, para cada $m\in\omega$, $f(2^m)=m$, mas $f$ não é injetiva, pois, por exemplo, $f(2^m)=f(3^m)$ e $2^m\neq 3^m$ se $m\geq1$.
\end{solucao}

\begin{exercicio}
	content
\end{exercicio}
\begin{solucao}
	content
\end{solucao}

\begin{exercicio}
	Seja $X$ um conjunto e sejam $x_0$ e $y_0$ dois elementos distintos de $X$. Considere a relação em $X$:
	$$R=\{(x,y)\in X\times X:x=y\}\cup\{(x_0,y_0),(y_0,x_0)\}$$
	\begin{enumerate}[label=(\alph{*})]
		\item Prove que $R$ é uma relação de equivalência em $X$.
		\item Descreva os elementos de $X/R$.
	\end{enumerate}
\end{exercicio}
\begin{solucao}
	\begin{enumerate}[label=(\alph{*})]
		\item Devemos demonstrar que a relação $R$ é reflexiva, simétrica e transitiva. Em linguagem de primeira ordem:
		$$xRy\leftrightarrow(x=y)\vee(x=x_0\wedge y=y_0)\vee(x=y_0\wedge y=x_0).$$
		
		\textbf{Reflexividade:} Para qualquer $x\in X$, $x=x$, logo $xRx$.
		
		\textbf{Simetria:} Se $xRy$, temos os seguintes casos:
		\begin{enumerate}[label=Caso \arabic{*}.]
			\item $x=y$: Então $y=x$, logo $yRx$;
			\item $x=x_0$: Então $y=x_0$ ou $y=y_0$. Se $y=x_0$, teremos $y=x$, logo $yRx$. Se $y=y_0$, o par $(x_0,y_0)\in R$ pela definição de $R$. Como $(y_0,x_0)\in R$ por definição, temos que vale $y_0Rx_0$, logo $yRx$.
			\item $x=y_0$: É análogo ao caso 2.
		\end{enumerate}
		
		\textbf{Transitividade:} Se $xRy$ e $yRz$, temos os casos
		\begin{enumerate}[label=Caso \arabic{*}.]
			\item $x\neq x_0$ e $y\neq y_0$: Então $x=y$ e $y=z$, logo $x=z$ e, portanto, $xRz$.
			\item $x=x_0$ e $y\neq y_0$: Se $x=x_0$, então $y=x_0$ ou $y=y_0$, mas pela hipótese deste caso, deve-se ter $y=x_0$. Como também $yRz$, então $y=z$ ou $y=y_0$ e $z=x_0$ (este caso está descartado), ou $y=x_0$ e $z=y_0$. Então $z=x_0$ (logo $zRx$ que, por reflexividade, dá $xRz$) ou $z=y_0$ (que, pela definição de $R$, $x_0Ry_0$, logo $xRz$). Em qualquer caso $xRz$, que desejávamos.
			\item $x\neq x_0$ e $y=y_0$: Análogo ao caso 2.
			\item $x=x_0$ e $y=y_0$: Como $yRz$, então $z=y_0$ ou $z=x_0$. Em qualquer caso $xRz$ pela definição de $R$.
		\end{enumerate}
		
		Isto conclui a prova de que $R$ é uma relação de equivalência.
		
		\item Lembre que $[x]_R=\{y\in X:xRy\}$. Dividimos novamente em casos.
		\begin{enumerate}[label=Caso \arabic{*}.]
			\item $x\neq x_0$ e $x\neq y_0$: Então $xRy\leftrightarrow x=y$, logo $[x]_R=\{x\}$.
			\item $x= x_0$ e $x\neq y_0$: Então $xRy\leftrightarrow (x=y)\vee (y=y_0)$, logo $[x]_R=\{x_0,y_0\}$.
			\item $x\neq x_0$ e $x= y_0$: Então $xRy\leftrightarrow (x=y)\vee (y=x_0)$, logo $[x]_R=\{x_0,y_0\}$.
		\end{enumerate}
		
		Em resumo, $X/R=\{\{x\}:x\in X\setminus\{x_0,y_0\}\}\cup\{\{x_0,y_0\}\}.$
	\end{enumerate}
\end{solucao}

\begin{exercicio}
	content
\end{exercicio}
\begin{solucao}
	content
\end{solucao}

\begin{exercicio}
	Como fica uma relação de equivalência sobre $\emptyset$? Ela satisfaz o Teorema 4.28?
\end{exercicio}
\begin{solucao}
	Se $R$ é uma relação sobre $\emptyset$, então $R\subset \emptyset\times\emptyset=\emptyset$, logo $R=\emptyset$. $\emptyset$ é uma relação de equivalência, já que não existem elementos de $\emptyset$ para desmentir esta afirmação. As classes de equivalência são todas vazias, ou melhor, $\forall x \in \emptyset ([x]_{\emptyset}=\emptyset)$. Isto mostra que o item (b) do Teorema 4.28 não vale. Portanto $\emptyset$ não satisfaz o teorema.
\end{solucao}

\begin{exercicio}
	content
\end{exercicio}
\begin{solucao}
	content
\end{solucao}

\begin{exercicio}
	Dê exemplo ou prove que não existe.
	\begin{enumerate}[label=(\alph{*})]
		\item Uma ordem total que não é uma boa ordem.
		\item Uma árvore que não é uma ordem total.
		\item Um reticulado que não é árvore.
		\item Uma árvore que é um reticulado mas não é totalmente ordenado.
	\end{enumerate}
\end{exercicio}
\begin{solucao}
	\begin{enumerate}[label=(\alph{*})]
		\item $(\mathbb{Q},\leq)$ é uma ordem total, mas não é bem ordenado, por exemplo o conjunto $\{x\in\mathbb{Q}:x>0\}$ não possui elemento mínimo.
		\item Considerando o conjunto $X=\{0,1,2\}$ e a ordem $\leq=\{(0,0),(1,1),(2,2),(0,1),(0,2)\}$, $(X,\leq)$ é uma árvore, mas não é uma ordem total -- por exemplo, não podemos comparar 1 com 2.
		\item O conjunto $X=\{0,1,2,3\}$ com a ordem
		$$\leq=\{(0,0),(1,1),(2,2),(3,3),(0,1),(0,2),(0,3),(1,3),(2,3)\}$$
		é um reticulado, mas não é uma árvore.
		\item Seja $(X,\leq)$ uma árvore que é também um reticulado. Dados $x,y\in X$, o conjunto $\{x,y\}$ possui um supremo, que chamaremos $s$. O conjunto $S=\{z\in X:z\leq s\}$ é bem ordenado. Então o conjunto $\{x,y\}\subset S$ possui um elemento mínimo, que deve ser $x$ ou $y$, logo $x\leq y$ ou $y\leq x$. Portanto $(X,\leq)$ é totalmente ordenado. Isto mostra que não existe uma árvore que é um reticulado mas não é totalmente ordenada.
	\end{enumerate}
\end{solucao}

\begin{exercicio}
	content
\end{exercicio}
\begin{solucao}
	content
\end{solucao}

\begin{exercicio}
	Prove que $1+1=2$.
\end{exercicio}
\begin{solucao}
	\begin{eqnarray*}
		1+1&=&s(1)(1)=s(1)(0^+)=(s(1)(0))^+=1^+ \\
		&=&1\cup\{1\}=\{0\}\cup\{\{0\}\}=\{0,\{0\}\} \\
		&=&\{0,1\} \\
		&=&2.
	\end{eqnarray*}
\end{solucao}

\begin{exercicio}
	content
\end{exercicio}
\begin{solucao}
	content
\end{solucao}

\begin{exercicio}
	Use o teorema da recursão e as operações de adição e multiplicação para definir cada uma das seguintes funções de domínio $\omega$. Em cada item, utilize a versão do teorema da recursão mais adequada.
	\begin{enumerate}[label=(\alph{*})]
		\item a função $f(n)=2^n$;
		\item o fatorial, isto é, $f(n)=n!$, onde $0!=1$ e $(n+1)!=(n+1)\cdot n!$;
		\item a sequência de Fibonacci, ou seja, a função $f:\omega\to\omega$ tal que $f(0)=f(1)=1$ e $f(n+2)=f(n)+f(n+1)$.
	\end{enumerate}
\end{exercicio}
\begin{solucao}
	\begin{enumerate}[label=(\alph{*})]
		\item Para cada $n$ podemos escrever $f(n+1)=2^{n+1}=2\cdot 2^n=2\cdot f(n)$. Isto nos motiva a definir $g:\omega\to\omega$ por $g(x)=2x$. Pelo Teorema da Recursão Finita, existe uma única função $f:\omega\to\omega$ tal que $f(0)=1$ e $f(n+1)=g(f(n))=2\cdot f(n)$.
		\item Para cada $n$ podemos escrever $f(n+1)=(n+1)!=(n+1)\cdot n!=(n+1)\cdot f(n)$. Isto nos motiva a definir $g:\omega\times\omega\to\omega$ por $g(x,y)=(x+1)\cdot y$. Pelo Teorema da Recursão Finita com Parâmetro, existe uma única função $f:\omega\to\omega$ tal que $f(0)=1$ e $f(n+1)=g(n,f(n))=(n+1)\cdot f(n)$.
		\item Defina $g:\omega^{<\omega}\to \omega$ por
		$$
		g(x)=\left\{
			\begin{array}{ll}
				1,&\mathrm{se}\ \dom x=0 \vee \dom x=1 \\
				x(n-1)+x(n-2),& \mathrm{se}\ \dom x=n>1.
			\end{array}
		\right.
		$$
		Pelo Teorema da Recursão Completa, existe uma única função $f:\omega\to\omega$ tal que $f(n)=g(f|n)$. Perceba que
		$$f(0)=g(f|0)=1 \qquad \mathrm{e} \qquad f(1)=g(f|1)=1,$$
		pois, pela definição de $g$, $g(x)=1$ se $\dom x=0$ ou $\dom x=1$. Para $n>1$, pela definição de $g$,
		$f(n)=g(f|n)=f(n-1)+f(n-2)$. Isto mostra que a função descrita no exercício existe de fato.
	\end{enumerate}
\end{solucao}
	\chapter{Axioma da Escolha e suas Aplicações}
	\include{capitulos/cap6}
	\include{capitulos/cap7}
	\include{capitulos/cap8}
	\include{capitulos/cap9}
	\chapter{No��es de Teoria dos Modelos}
	\include{capitulos/cap11}
	\include{capitulos/cap12}
	
	%\bibliographystyle{amsplain}
	%\printbibliography[title=Refer�ncias]
	%\bibliography{bibliografia.bib}
	\begin{thebibliography}{10}
		\bibitem[Fa]{Fajardo} FAJARDO, R. A. dos S. \textit{A Teoria dos Conjuntos e os Fundamentos da Matem�tica}. S�o Paulo: Edusp, 2024.
		\bibitem[Li]{Lima16} LIMA, E. L. \textit{Curso de an�lise}. 14 ed. Rio de Janeiro: Impa, 2016. v. 1.
	\end{thebibliography}
\end{document}