\documentclass[portuguese,12pt,a4paper]{book}
\usepackage[utf8]{inputenc}
\usepackage[T1]{fontenc}
\usepackage{graphicx}
\usepackage{mathtools}
\usepackage{amssymb}
\usepackage{amsthm}
\usepackage{babel}
\usepackage[colorlinks=true]{hyperref}
\usepackage{enumitem}
%\usepackage[
%	backend=bibtex,
%	style=alphabetic,
%	sorting=ynt
%]{biblatex}
%\addbibresource{bibliografia.bib}

\title{A teoria dos conjuntos e os fundamentos da matemática - Soluções}

\newcommand{\cqd}{\hfill $\square$}
\newenvironment{solucao}[1][]{\noindent\textbf{Solução#1:} }{\cqd}
\newcounter{ex}
\newtheorem{exercicio}[ex]{Exercício}
\DeclareMathOperator{\dom}{dom}

\begin{document}
	
	\maketitle
	\tableofcontents
	
	\chapter{História e Motivação}

\begin{exercicio}
	Exiba uma bijeção entre o conjunto dos números inteiros e os naturais.
\end{exercicio}
\begin{solucao}
	Defina a função $f:\mathbb{N}\to\mathbb{Z}$ por $f(n)=-\frac{n}{2}$ se $n$ é par e por $f(n)=\frac{n+1}{2}$ se $n$ é ímpar. $f$ é uma bijeção.
\end{solucao}

\begin{exercicio}
	Prove que qualquer subconjunto infinito dos números naturais é enumerável.
\end{exercicio}
\begin{solucao}
	Lembre: dizemos que um conjunto $X$ é \textit{enumerável} se é finito ou é equipotente a $\mathbb{N}$, isto é, se existe uma bijeção $f:\mathbb{N}\to X$.
	
	Todo conjunto finito é enumerável. Estão vamos olhar apenas o caso $X$ ser infinito. Defina $f:\mathbb{N}\to X$ da seguinte maneira: $f(0)$ é o menor elemento de $X$; $f(1)$ é o segundo menor elemento de $X$; $f(2)$ é o terceiro menor elemento de $X$, e assim por diante. De maneira mais rigorosa, $f$ é definida indutivamente por: $f(0)=\min X$; uma vez escolhidos, $f(0),f(1),\ldots,f(n)$, definimos $f(n+1)=\min(X\backslash\{f(0),f(1),\ldots,f(n)\})$. Observe que $f$ é crescente, logo injetiva. Também é sobrejetiva, pois do contrário, se existe um $x\in X\backslash f(\mathbb{N})$, então $x\in X\backslash\{f(0),f(1),\ldots,f(n)\}$ para cada $n\in\mathbb{N}$. Da maneira como construímos $f$ concluímos que $x>f(n)\ \forall n\in\mathbb{N}$ e assim $f(\mathbb{N})$ é um conjunto limitado, logo finito. Como $f:\mathbb{N}\to f(\mathbb{N})$ é uma bijeção, $\mathbb{N}$ seria finito, um absurdo. Portanto $f:\mathbb{N}\to X$ é uma bijeção, logo $X$ é enumerável.
\end{solucao}

\begin{exercicio}
	Na bijeção que construímos entre os números naturais e os polinômios, encontre o polinômio associado ao número 30.
\end{exercicio}
\begin{solucao}
	\begin{equation*}
		\begin{array}{|ccc|ccc|ccc|}
			\hline
			0  & \mapsto & -x-1  & 11 & \mapsto & -x-2       & 21 & \mapsto & -2x^2-2x-1 \\ \hline
			1  & \mapsto & -x    & 12 & \mapsto & -x+2       & 22 & \mapsto & -2x^2-2x   \\ \hline
			2  & \mapsto & -x+1  & 13 & \mapsto & x-2        & 23 & \mapsto & -2x^2-2x+1 \\ \hline
			3  & \mapsto & x-1   & 14 & \mapsto & x+2        & 24 & \mapsto & -2x^2-2x+2 \\ \hline
			4  & \mapsto & x     & 15 & \mapsto & 2x-2       & 25 & \mapsto & -2x^2-x-2  \\ \hline
			5  & \mapsto & x+1   & 16 & \mapsto & 2x-1       & 26 & \mapsto & -2x^2-x-1  \\ \hline
			6  & \mapsto & -2x-2 & 17 & \mapsto & 2x         & 27 & \mapsto & -2x^2-x    \\ \hline
			7  & \mapsto & -2x-1 & 18 & \mapsto & 2x+1       & 28 & \mapsto & -2x^2-x+1  \\ \hline
			8  & \mapsto & -2x   & 19 & \mapsto & 2x+2       & 29 & \mapsto & -2x^2-x+2  \\ \hline
			9  & \mapsto & -2x+1 & 20 & \mapsto & -2x^2-2x-2 & 30 & \mapsto & -2x^2-2    \\ \hline
			10 & \mapsto & -2x+2 &    &         &            &    &         &            \\ \hline
		\end{array}
	\end{equation*}
	Portanto o polinômio associado ao número 30 é $-2x^2-2$.
\end{solucao}

\begin{exercicio}
	Na bijeção que construímos entre os números naturais e os números algébricos, encontre o número natural associado ao número $\sqrt{3}$
\end{exercicio}
\begin{solucao}
	content
\end{solucao}

\begin{exercicio}
	Suponha que, em um conjunto infinito, existe uma forma de representar cada elemento do conjunto com uma sequência finita de símbolos, dentre um conjunto finito de símbolos. Mostre que esse conjunto é enumerável e use esse resultado diretamente para mostrar que os conjuntos dos números racionais e dos números algébricos são enumeráveis.
\end{exercicio}
\begin{solucao}
	Seja $\Sigma=\{\sigma_1,\ldots,\sigma_m\}$ o conjunto de símbolos. Para cada $n\in\mathbb{N}$, seja $S_n=\{(\tau_1,\ldots,\tau_n):\tau_1,\ldots,\tau_n\in \Sigma\}$ e ponha $S=\bigcup_{n=1}^\infty S_n$. $S$ é enumerável por ser uma reunião enumerável de conjuntos enumeráveis (finitos).\footnote{\cite[p. 51]{Lima16}} Seja $X$ o conjunto infinito do exercício. A suposição de que cada elemento de $X$ possa ser representado como uma sequência finita de símbolos significa que existe uma função $f:X\to S$. Esta função deve ser injetiva, pois não podemos usar a mesma sequência de símbolos para representar dois ou mais elementos distintos em $X$. Então $f: X\to f(X)\subset S$ é bijetiva e como $f(X)$ é enumerável, por ser um subconjunto de um conjunto enumerável conforme o Exercício 2, $X$ é enumerável.
	
	Todo número racional pode ser escrito como $a/b$, com $a,b\in \mathbb{Z}$, $b\neq 0$. Ambos $a$ e $b$ são escritos como uma sequência finita de símbolos (os algarismos de 0 a 9 e o sinal $+$ ou $-$). Pelo resultado demonstrado acima, $\mathbb{Q}$ é enumerável.
	
	Cada polinômio com coeficientes inteiros pode ser escrito como uma sequência finita de símbolos, logo o conjunto formado por esses polinômios é enumerável. O conjunto dos números algébricos é a união dos conjuntos das raízes de cada um desses polinômios. Como essa é uma união enumerável (o conjunto desses polinômios é enumerável) de conjuntos finitos (cada polinômio possui um número finito de raízes), o conjunto dos números algébricos é enumerável.
\end{solucao}

\begin{exercicio}
	Imagine que o hotel de Hilbert, com uma quantidade infinita enumerável de quartos, todos ocupados, receba infinitos trens com infinitos vagões e cada vagão com infinitos passageiros (todas essas quantidades enumeráveis). Como o gerente pode alocar todos os atuais hóspedes em quartos separados?
\end{exercicio}
\begin{solucao}
	Cada passageiro tem uma identificação $(p,q,r)$: $p$ = número do trem; $q$ = número do vagão; $r$ = número do assento. Cada hóspede no quarto $n$ será realocado para o quarto $2n-1$, e cada passageiro com ID $(p,q,r)$ será hospedado no quarto $2^p3^q(2r-1)$.
	
	Isto pode ser generalizado: hóspede $n\mapsto$ quarto $2n-1$; passageiro $(a_1,\ldots,a_m)\mapsto$ quarto $2^{a_1}3^{a_2}\cdots p_{m-1}^{a_{m-1}}(2a_m-1)$, em que $2,3,\ldots,p_{m-1}$ são números primos.
\end{solucao}

\begin{exercicio}
	Imagine, agora, um hotem maior ainda, com um quarto para cada número real, totalmente ocupado. Um ônibus igualmente gigantesco, com um passageiro para cada número real, chega ao hotel. Como o gerente pode fazer para rearranjar os hóspedes para acolher os novos visitantes, sempre em quartos separados?
\end{exercicio}
\begin{solucao}
	\begin{itemize}
		\item Hóspede $x\mapsto$ quarto $\arctan x$.
		\item Passageiro $y\mapsto$ quarto $y+\pi/2$ se $y\geq0$ ou $y-\pi/2$ se $y<0$. 
	\end{itemize}
	Note que o hotel ficará com um quarto vago (o de número $-\pi/2$).
\end{solucao}
	\chapter{A linguagem da Teoria dos Conjuntos}

\setcounter{ex}{0}

\begin{exercicio}
	Usando a linguagem de primeira ordem da teoria de conjuntos, escreva fórmulas para representar as seguintes frases.
	\begin{enumerate}[label=(\alph{*})]
		\item Não existe o conjunto de todos os conjuntos.
		\item Existe um único conjunto vazio.
		\item x é um conjunto unitário
		\item Existe um conjunto que tem como elemento apenas o conjunto vazio
		\item y é o conjunto dos subconjuntos de x 
	\end{enumerate}
\end{exercicio}

\begin{solucao}
	\begin{enumerate}[label=(\alph{*})]
		\item $\neg \exists x \forall y (y \in x)$ ou $\forall x \exists y (y \notin x)$.
		\item $\exists ! x \forall y (y \notin x)$.
		\item $\exists ! y (y \in x)$.
		\item $\exists x \forall y ((y \in x) \leftrightarrow y = \phi)$.
		\item $\forall z (z \in y \leftrightarrow \forall w (w \in z \to w \in x))$.
	\end{enumerate}
\end{solucao}

\begin{exercicio}
	Marque as ocorrências de variáveis livres nas fórmulas abaixo
\end{exercicio}
\begin{enumerate}[label=(\alph{*})]
	\item $(\forall x (x=y)) \rightarrow (x \in y ) $
	\item $ \forall x ((x=y) \rightarrow (x \in y))$
	\item $\forall x(x=x) \rightarrow (\forall y \exists Z ((x=y) \land (y=z)) \rightarrow \neg(x\in y))$
	\item $ \forall x \exists y(\neg(x=y) \land \forall z ((x \in y) \leftrightarrow \forall w ((w \in z ) \rightarrow (w \in x )))) $
	\item $(x=y)\rightarrow \exists (x=y) $
\end{enumerate}

\begin{solucao}
	\begin{enumerate}[label=(\alph{*})]
    \item $x$ e $y$
    \item $y$ 
    \item $x$
    \item Não há variáveis livres.
    \item $x$ e $y$
\end{enumerate}
\end{solucao}

\begin{exercicio}
	Escreva as subfórmulas de cada fórmula do exercício 2.
\end{exercicio}

\begin{solucao}

\begin{enumerate}[label=(\alph{*})]
	\item 
	\begin{itemize}
		\item $(\forall x (x = y)) \rightarrow (x \in y)$
		\item $(\forall x (x = y))$
		\item $(x = y)$
		\item $(x \in y)$
	\end{itemize}
	
	\item 
	\begin{itemize}
		\item $\forall x ((x = y) \rightarrow (x \in y))$
		\item $(x = y) \rightarrow (x \in y)$
		\item $(x = y)$
		\item $(x \in y)$
	\end{itemize}
	
	\item 
	\begin{itemize}
		\item $\forall x (x = x) \rightarrow (\forall y \exists z (((x = y) \land (y = z)) \rightarrow \neg (x \in y)))$
		\item $\forall x (x = x)$
		\item $(x = x)$
		\item $\forall z \exists y (((x = y) \land (y = z)) \rightarrow \neg (x \in y))$
		\item $((x = y) \land (y = z))$
		\item $(x = y)$
		\item $(y = z)$
		\item $\neg (x \in y)$
		\item $(x \in y)$
	\end{itemize}
	
	\item 
	\begin{itemize}
		\item $(x = y) \rightarrow \exists y (x = y)$
		\item $(x = y)$
		\item $\exists y (x = y)$
	\end{itemize}
\end{enumerate}

\end{solucao}
	\chapter{Primeiros Axiomas}

\setcounter{ex}{0}

\begin{exercicio}
	Usando o axioma da extens�o, prove que $\{\emptyset\}$ e $\{\{\emptyset\}\}$ s�o conjuntos diferentes.
\end{exercicio}
\begin{solucao}
	O conjunto $\{\emptyset\}$ tem como �nico elemento $\emptyset$, enquanto $\{\{\emptyset\}\}$ tem como �nico elemento $\{\emptyset\}$. Como $\emptyset\neq\{\emptyset\}$, segue do axioma da extens�o que $\{\emptyset\}$ e $\{\{\emptyset\}\}$ s�o conjuntos distintos.
\end{solucao}

\begin{exercicio}
	Para cada par de conjuntos abaixo, decida qual(is) dos s�mbolos $\in$ e $\subset$ torna(m) a f�rmula verdadeira (assumindo que esses conjuntos existem). Lembre-se de que a resposta tamb�m pode ser ambos os s�mbolos ou nenhum deles. Justifique cada resposta.
	\begin{enumerate}[label=(\alph{*})]
		\item $\{\emptyset\}\ldots\{\emptyset,\{\emptyset\}\}$
		\item $\{\emptyset\}\ldots\{\{\emptyset\}\}$
		\item $\{1,2,3\}\ldots\{\{1\},\{2\},\{3\}\}$
		\item $\{1,2,3\}\ldots\{\{1\},\{1,2\},\{1,2,3\}\}$
		\item $\{1,2\}\ldots\{1,\{1\},2,\{2\},\{3\}\}$
		\item $\{\{1\},\{2\}\}\ldots\{\{1,2\}\}$
	\end{enumerate}
\end{exercicio}

\begin{enumerate}[label=(\alph{*})]
	\item $\{\emptyset\}\in\{\emptyset,\{\emptyset\}\}$ e $\{\emptyset\}\subset\{\emptyset,\{\emptyset\}\}$.
	\item $\{\emptyset\}\in\{\{\emptyset\}\}$.
	\item N�o vale $\in$ nem $\subset$.
	\item $\{1,2,3\}\in\{\{1\},\{1,2\},\{1,2,3\}\}$.
	\item $\{1,2\}\subset\{1,\{1\},2,\{2\},\{3\}\}$.
	\item N�o vale $\in$ nem $\subset$.
\end{enumerate}


\begin{exercicio}
	Seja $x$ o conjunto $\{\emptyset,\{\emptyset\},\emptyset,\{\emptyset,\{\emptyset\}\}\}$.
	\begin{enumerate}[label=(\alph{*})]
		\item Quantos elementos tem o conjunto $x$?
		\item Descreva todos os subconjuntos de $x$.
		\item Descreva, usando chaves e v�rgula, o conjunto de todos os subconjuntos de $x$.
		\item Quantos elementos o conjunto dos subconjuntos de $x$ possui?
		\item Prove que o conjunto $x$ existe.
	\end{enumerate}
\end{exercicio}

\begin{enumerate}[label=(\alph{*})]
	\item 3 elementos.
	\item $\emptyset,\{\emptyset\},\{\{\emptyset,\{\emptyset\}\}\},\{\emptyset,\{\emptyset\}\},\{\{\emptyset,\{\emptyset,\{\emptyset\}\}\}\},\{\emptyset,\{\emptyset,\{\emptyset\}\}\},\{\emptyset,\{\emptyset\},\{\emptyset,\{\emptyset\}\}\}\}$.
	\item $\{\emptyset,\{\emptyset\},\{\{\emptyset,\{\emptyset\}\}\},\{\emptyset,\{\emptyset\}\},\{\{\emptyset,\{\emptyset,\{\emptyset\}\}\}\},\{\emptyset,\{\emptyset,\{\emptyset\}\}\},\{\emptyset,\{\emptyset\},\{\emptyset,\{\emptyset\}\}\}\}\}$.
	\item 8 elementos.
	\item 
	\begin{description}
		\item[Solu��o 1:] Pelo axioma do vazio, existe $\emptyset$. Pelo axioma das partes, existe $\mathcal{P}(\emptyset)=\{\emptyset,\{\emptyset\}\}$, e ent�o $x=\mathcal{P}(\emptyset)\cup\{\mathcal{P}(\emptyset)\}=\{\emptyset,\{\emptyset\},\{\emptyset,\{\emptyset\}\}\}$ pelo Teorema 3.7. $\qed$
		\item[Solu��o 2:] Peguemos do conjunto $\omega$, da defini��o 3.19, o elemento que � representante do n�mero tr�s, ou seja, $y = \{\emptyset, \{\emptyset\}, \{\emptyset, \{\emptyset\}\}\} \in \omega$. Pelo Axioma da Extens�o, $x = y$. $\qed$ 
	\end{description}
\end{enumerate}

\begin{exercicio}
	Prove que para todos conjuntos $x$ e $y$
	\begin{enumerate}[label=(\alph{*})]
		\item $x\subset x$;
		\item $x\in y$ se, e somente se, $\{x\}\subset y$;
		\item $\bigcup\mathcal{P}(x)=x$;
		\item se $x\subset y$, ent�o $\bigcup x\subset \bigcup y$.
	\end{enumerate}
\end{exercicio}

\begin{enumerate}[label=(\alph{*})]
	\item 
		\begin{solucao} 
			Se $x$ n�o � vazio, todo elemento de $x$ � elemento de $x$, logo $x\subset x$. Se $x$ � vazio, como $\emptyset$ est� contido em qualquer conjunto pelo Teorema 3.4, temos que $\emptyset\subset \emptyset$. Em qualquer caso, $x\subset x$. 
		\end{solucao}
	\item 
		\begin{solucao}
			Se $x\in y$, como $x$ � o �nico elemento do conjunto $\{x\}$, temos que $\{x\}\subset y$. Reciprocamente, se $\{x\}\subset y$, ent�o $x\in y$.
		\end{solucao}
	\item 
		\begin{solucao}
			Dado $u\in\bigcup\mathcal{P}(x)$, existe $v\in\mathcal{P}(x)$ tal que $u\in v$. Como $\mathcal{P}(x)$ � o conjunto de todos os subconjuntos de $x$, $v\subset x$, logo $u\in x$. Portanto $\bigcup\mathcal{P}(x)\subset x$.
			
			Reciprocamente, como $x\subset x$ pelo item (a), ent�o $x\in\mathcal{P}(x)$, logo qualquer elemento $y\in x$ ser� elemento de $\bigcup\mathcal{P}(x)$, e assim $x\subset \bigcup\mathcal{P}(x)$.
			
			Pelo axioma da extens�o, $\bigcup\mathcal{P}(x)=x$.
		\end{solucao}
	\item 
		\begin{solucao}
			Dado $u\in\bigcup x$, existe $v\in x$ tal que $u\in v$. Como $x\subset y$, $v\in y$, logo $u\in\bigcup y$ e, portanto, $\bigcup x\subset\bigcup y$.
		\end{solucao}
\end{enumerate}

\begin{exercicio}
	Escreva uma f�rmula de primeira ordem, na linguagem da teoria dos conjuntos, com quatro vari�veis livres, que represente o conjunto $\{x,y,z\}$.
\end{exercicio}
\begin{solucao}
	$\forall w((w\in u)\leftrightarrow((w=x)\vee(w=y)\vee(w=z)))$.
\end{solucao}

\begin{exercicio}
	Escreva os seguintes conjuntos, listando seus elementos entre chaves:
	\begin{enumerate}[label=(\alph{*})]
		\item $\bigcup\{\{0,1\},\{\{1\}\},\{1,2\},\{\{1,2\}\}\}$;
		\item $\mathcal{P}(\{\emptyset,\{\emptyset\}\})$.
	\end{enumerate}
\end{exercicio}

\begin{enumerate}[label=(\alph{*})]
	\item $\{0,1,\{1\},2,\{1,2\}\}$.
	\item $\{\emptyset,\{\emptyset\},\{\{\emptyset\}\},\{\emptyset,\{\emptyset\}\}\}$.
\end{enumerate}


\begin{exercicio}
	Prove que n�o existe o conjunto de todos os conjuntos unit�rios.
	
	\emph{Dica:} Assuma, por absurdo, a exist�ncia do conjunto de todos os conjuntos unit�rios e prove a exist�ncia do conjunto de todos os conjuntos.
\end{exercicio}

\begin{proof}[\textbf{Solu��o 1}:]
	Suponha que existe $x$ tal que $\forall y(\{y\}\in x)$. Ent�o $\{x\}\in x$. Como $x\in\{x\}$, temos uma contradi��o com o Teorema 3.14. Portanto $\nexists x\forall y(\{y\}\in x)$.
\end{proof}
	
\begin{proof}[\textbf{Solu��o 2}:]
	Suponha, por absurdo, que $u$ seja o conjunto de todos os conjuntos unit�rios. Seja $x$ um conjunto arbitr�rio, ent�o $\{x\} \in u$ por hip�tese e, pelo Axioma da Uni�o, teremos $x \in \bigcup x$. Isso faz de $\bigcup x$ o conjunto de todos os conjuntos, o que � um absurdo pelo Teorema 3.10.
\end{proof}
	

\begin{exercicio}
	Prove que para todo conjunto $X$ existe o conjunto
	$$\{\{x\}:x\in X\}$$
\end{exercicio}
\begin{solucao}
	Pelo axioma das partes, existe o conjunto $\mathcal{P}(X)$. Para cada $x\in X$, temos que $\{x\}\subset X$, logo $\{x\}\in\mathcal{P}(X)$. Pelo axioma da separa��o, existe o conjunto $\{y\in \mathcal{P}(X):\exists x((x\in X)\wedge(\{x\}=y))\}$, que, via axioma da extens�o, � o conjunto procurado.
\end{solucao}

\begin{exercicio}
	Sendo $x$ um conjunto n�o vazio, prove que
	\begin{enumerate}[label=(\alph{*})]
		\item $\forall y(y\in x\to(\bigcap x\subset y))$;
		\item $x\subset y\to \bigcap y\subset \bigcap x$.
	\end{enumerate}
\end{exercicio}

\begin{enumerate}[label=(\alph{*})]
	\item
		\begin{description}
			\item[Solu��o 1] Dados $y\in x$ e $z\in\bigcap x$, temos que $\forall w((w\in x)\to(z\in w))$. Em particular, $z\in y$, logo $\bigcap x\subset y$. $\qed$
			\item[Solu��o 2] Seja $x$ um conjunto qualquer n�o vazio e $y$ um de seus elementos. Se $\bigcap x = \emptyset$ ent�o $\bigcap x \subset y$ pelo Teorema 3.4. Caso contr�rio, escolha arbitrariamente um elemento $z$ de $\bigcap x$, ent�o $z$ pertence a todos os elementos de $x$, desse modo, teremos $z \in y$ e, novamente, $\bigcap x \subset y$. $\qed$
		\end{description}
	\item Dado $z\in\bigcap y$, temos que $z$ � elemento de qualquer $w\in y$. Como $x\subset y$, os elementos de $x$ s�o tamb�m elementos de  $y$, logo $z$ � elemento de qualquer $w\in x$ em particular, portanto $z\in\bigcap x$. Desta forma $\bigcap y\subset\bigcap x$.
\end{enumerate}


\begin{exercicio}
	Escreva na linguagem da l�gica de primeira ordem, sem abreviaturas, a seguinte f�rmula:
	$$x\in\bigcup\bigcap(y\cup(w\backslash z)).$$
\end{exercicio}
\begin{solucao}
	Faremos por partes. Dizer que $x\in\bigcup a$ para algum conjunto $a$ significa que $\exists u((x\in u)\wedge(u\in a))$. Chame $a=\bigcap b$, sendo $b$ um conjunto n�o vazio. Ent�o $x\in a\leftrightarrow\forall v((v\in b)\rightarrow(x\in v))$. Desta forma, $x\in \bigcup\bigcap b$ significa
	$$\exists u((x\in u)\wedge\forall v((v\in b)\rightarrow(u\in v))).$$
	Fa�a $b=y\cup c$. Ent�o $x\in c\leftrightarrow(x\in y)\vee(x\in c)$. Assim
	$$x\in \bigcup\bigcap (y\cup c)\leftrightarrow\exists u((x\in u)\wedge\forall v(((v\in y)\vee(v\in c))\rightarrow(u\in v))).$$
	Por fim, fa�a $c=w\backslash z$. Ent�o $x\in c\leftrightarrow(x\in w)\wedge\neg(x\in z)$. Portanto $x\in \bigcup\bigcap (y\cup(w\backslash z))$ significa
	$$\exists u((x\in u)\wedge\forall v(((v\in y)\vee((v\in w)\wedge\neg(v\in z)))\rightarrow(u\in v))).$$
\end{solucao}

\begin{exercicio}
	Usando o axioma da regularidade, prove que:
	\begin{enumerate}[label=(\alph{*})]
		\item n�o existem $x,y,z$ tais que $x\in y,y\in z$ e $z\in x$;
		\item n�o existem $w,x,y,z$ tais que $w\in x,x\in y,y\in z$ e $z\in w$.
	\end{enumerate}
\end{exercicio}
\begin{solucao}
	\begin{enumerate}[label=(\alph{*})]
		\item Pelo axioma do par, existem $\{x,y\}$ e $\{y,z\}$. Tome $w=\{x,y\}\bigcup \{y,z\}=\{x,y,z\}$. Pelo axioma da regularidade, existe $u\in w$ tal que $u\cap w=\emptyset$. Se for $u=x$, ent�o $y\notin x$ e $z\notin x$. Se for $u=y$, ent�o $x\notin y$ e $z\notin y$. Se for $u=z$, ent�o $x\notin z$ e $y\notin z$. Em qualquer caso, n�o podemos ter $x\in y$, $y\in z$ e $z\in x$ simultaneamente.
		\item Basta repetir o argumento acima para o conjunto $\{w,x,y,z\}$.
	\end{enumerate}
\end{solucao}

\begin{exercicio}
	Prove que n�o existe $x$ tal que $\mathcal{P}(x)=x$.
\end{exercicio}
\begin{solucao}[ 1]
	Se $\mathcal{P}(x)=x$, como $x\subset x$, ent�o $x\in \mathcal{P}(x)$, contradizendo o Corol�rio 3.15.
\end{solucao}

\begin{solucao}[ 2]
	Suponha que $x$ seja um conjunto tal que $\mathcal{P}(x)=x$. Seja $y$ um elemento qualquer de $x$, ent�o tamb�m teremos $y \subset x$, dado que assumimos $\mathcal{P}(x)=x$, e portanto $x \cap y \neq \emptyset$, o que � um absurdo pelo Axioma da Regularidade.
\end{solucao}

\begin{exercicio}
	Escreva o conjunto $\mathcal{P}(3\backslash 1)$, utilizando apenas os seguintes s��mbolos: as chaves, a v��rgula e o s��mbolo de conjunto vazio.
\end{exercicio}
\begin{solucao}
	$\mathcal{P}(3\backslash 1)=\{\emptyset,\{\emptyset\},\{\{\emptyset,\{\emptyset\}\}\},\{\emptyset,\{\emptyset,\{\emptyset\}\}\}\}$.
\end{solucao}

\begin{exercicio}
	Prove, a partir dos axiomas de Peano, os seguintes teoremas:
	\begin{enumerate}[label=(\alph{*})]
		\item Todo n�mero natural � diferente do seu sucessor.
		\item Zero � o �nico n�mero natural que n�o � sucessor de nenhum n�mero natural.
	\end{enumerate}
\end{exercicio}

\begin{enumerate}[label=(\alph{*})]
	\item
	\begin{solucao}[ 1]
		Tese: $\forall n((n\in\omega)\wedge(n\neq n^+))$.
		
		Para $n=0$, $n^+\neq0$, pois $0$ n�o � sucessor de nenhum n�mero natural, logo a tese � verdadeira para $n=0$.
		
		Suponha que a tese � verdadeira para algum $n\in\omega$. Pelo axioma 3, $n$ e $n^+$ devem ter sucessores distintos, isto �, $n^+\neq(n^+)^+$. Isto mostra que a tese � verdadeira para $n^+$. Pelo axioma 5, a tese � verdadeira para todo $n$ natural.
	\end{solucao}
	
	\begin{solucao}[ 2]
			Usemos $\omega$ da defini��o 3.19 como modelo para $\mathbb{N}$. Pelo Teorema 3.20, $\omega$ satisfaz os Axiomas de Peano e, portanto, podemos aplicar o Princ��pio da Indu��o Finita. Por indu��o, verifiquemos que:
			\begin{description}
				\item[\textbf{Caso Base}:] Pelo Axioma da Extens�o temos que $\emptyset \neq (\emptyset \cup \{\emptyset\}) = \{\emptyset\} = \emptyset^+$.
				\item[\textbf{Passo indutivo}:] Seja $x$ um elemento qualquer de $\omega$ e suponha que $x \neq x^+$, ent�o: \newline \newline $(x^+)^+ = x^+ \cup \{x^+\}$ \newline $=  (x \cup \{x\}) \cup \{x^+\}$ \newline $= x \cup (\{x\} \cup \{x^+\})$ \newline $= x \cup \{x, x^+\} \neq x \cup \{x\} = x^+$ \newline \newline Com isso garantimos, pelo quinto Axioma de Peano, que todo n�mero natural � diferente de seu sucessor.
			\end{description}
		\end{solucao}
	
	\item  
	\begin{solucao}[ 1]
		
		\textbf{Tese:}
			\[
			\forall n \big((n \in \omega) \wedge \nexists m \big((m \in \omega) \wedge (m^+ = n)\big) \to (n = 0)\big).
			\]
			
			\textbf{Contrapositiva:}  
			\[
			\forall n \big((n \neq 0) \to ((n \notin \omega) \vee \exists m \big((m \in \omega) \wedge (m^+ = n)\big))\big).
			\]
			
			Demonstraremos a contrapositiva, por ser equivalente � tese. Para isso, precisaremos da seguinte proposi��o:
			
			\vspace{0.3cm}
			\textbf{Proposi��o:} Seja $\sigma$ um subconjunto n�o vazio de $\omega$. Se $0 \in \sigma$ e, para cada $n \in \sigma$, valer $n^+ \in \sigma$, ent�o $\sigma = \omega$.
			
			\vspace{0.3cm}
			\textbf{Prova da Proposi��o:} Aplicando o axioma 5 � f�rmula $P(\sigma) = (x \in \sigma)$, obtemos $\omega \subset \sigma$. Como $\sigma \subset \omega$, ent�o $\sigma = \omega$.  
			
			Seja 
			\[
			\sigma = \{0\} \cup \{n \in \omega : \exists m ((m \in \omega) \wedge (m^+ = n))\}.
			\]
			Temos que $0 \in \sigma$ e, para cada $n \in \sigma$, $n^+ \in \sigma$. Pela proposi��o acima, $\sigma = \omega$.  
			
			Pelo axioma 4, 
			\[
			\{0\} \cap \{n \in \omega : \exists m ((m \in \omega) \wedge (m^+ = n))\} = \emptyset.
			\]
			Portanto, para qualquer $n \in \omega$ com $n \neq 0$,  
			\[
			\exists m ((m \in \omega) \wedge (m^+ = n)).
			\]
			Isto demonstra, por contraposi��o, a tese.
	\end{solucao}
	
	\begin{solucao}[ 2]
		Seja $n$ um elemento de $\omega$ tal que $n \neq \emptyset$ e suponha que $n$ n�o � sucessor de nenhum outro n�mero natural de $\omega$. Ent�o, pelo Teorema 3.21, item $e$, temos os seguintes cen�rios:
		
		\begin{enumerate}[label=\textit{\arabic*� --}, left=0pt, itemsep=0.5em]
			\item $\emptyset = n$: o que contradiz nossa hip�tese.
			\item $\emptyset \in n$: ent�o, pelo item $b$ do Teorema 3.21, teremos os seguintes subcasos:
			\begin{itemize}
				\item $\emptyset^+ = n$: o que novamente contradiz nossa hip�tese.
				\item $\emptyset^+ \in n$: ent�o $n$ existe gra�as a sucessivas aplica��es\footnote{A ideia de aplicar uma mesma propriedade de forma sucessiva ser� formalizada no Cap�tulo 4 com o Teorema da Recurs�o.} da \textbf{Defini��o 3.16}, e h� um 
				\[
				m = (\cdots((\emptyset^+)^+)^+\cdots)^+
				\]
				tal que $m^+ = n$, contrariando nossa hip�tese.
			\end{itemize}
			\item $n \in \emptyset$: o que � um absurdo.
		\end{enumerate}
		
		Como esgotamos todos os cen�rios e, em todos eles, chegamos a uma contradi��o, n�o pode haver $n$, al�m de $\emptyset$, em $\omega$ que n�o seja sucessor de nenhum outro n�mero natural.
	\end{solucao}
\end{enumerate}

\begin{exercicio}
	Prove que:
	\begin{enumerate}[label=(\alph{*})]
		\item para todo $n\in\omega$, $\emptyset\in n$ ou $\emptyset=n$;
		\item para todos $n,m\in\omega$, se $m\in n$, ent�o $m\subset n$.
	\end{enumerate}
\end{exercicio}

\begin{enumerate}[label=(\alph{*})]
	\item Para $n=0$, $n=\emptyset$, logo a tese � verdadeira.
	
	Suponha ser verdade para $n$. Se $\emptyset=n$, ent�o $\emptyset\in n^+=n\cup\{n\}$. Se $\emptyset\in n$, ent�o $\emptyset\in n^+$, e a tese � verdadeira para $n^+$
	
	Pelo axioma 5, $\emptyset\in n$ ou $\emptyset=n$ para qualquer $n\in\omega$.
	\item J� feito na prova do Teorema 3.21, item (c).
\end{enumerate}


\begin{exercicio}
	A uni�o de dois conjuntos indutivos � necessariamente um conjunto indutivo? Justifique sua resposta.
\end{exercicio}
\begin{solucao}
	
	\underline{Resposta:} Sim.
	
	\underline{Justificativa:} Sejam $A$ e $B$ dois conjuntos indutivos. Como $\emptyset\in A$ e $\emptyset\in B$, ent�o $\emptyset\in A\cup B$. Se $x\in A\cup B$, ent�o $x\in A$ ou $x\in B$, logo $x^+\in A$ ou $x^+\in B$, o que implica $x^+\in A\cup B$. Isto mostra que $A\cup B$ � indutivo.
\end{solucao}
	\chapter{Produto Cartesiano, Relações e Funções}

\setcounter{ex}{0}

\begin{exercicio}
	content
\end{exercicio}
\begin{solucao}
	content
\end{solucao}

\begin{exercicio}
	Prove que, se $A\subset C$ e $B\subset D$, então $A\times B\subset C\times D$.
\end{exercicio}
\begin{solucao}
	Dado um par ordenado $(a,b)\in A\times B$, temos que $a\in A$ e $b\in B$. Uma vez que $A\subset C$ e $B\subset D$, então $a\in C$ e $b\in D$, logo $(a,b)\in C\times D$. Portanto $A\times B\subset C\times D$.
\end{solucao}

\begin{exercicio}
	content
\end{exercicio}
\begin{solucao}
	content
\end{solucao}

\begin{exercicio}
	Descreva todos os elementos de $\mathcal{P}(2\times 2)$.
\end{exercicio}
\begin{solucao}
	$$2=\{0,1\}\Rightarrow 2\times2=\{(0,0),(0,1),(1,0),(1,1)\}$$
	\begin{eqnarray*}
		\Rightarrow\mathcal{P}(2\times 2)&=&\{\emptyset,\{(0,0)\},\{(0,1)\},\{(1,0)\},\{(1,1)\},\{(0,0),(0,1)\}, \\
		&&\{(0,0),(1,0)\},\{(0,0),(1,1)\},\{(0,1),(1,0)\},\{(0,1),(1,1)\} \\
		&&\{(0,0),(0,1),(1,0)\},\{(0,0),(0,1),(1,1)\},\{(0,0),(1,0),(1,1)\} \\
		&&\{(0,1),(1,0),(1,1)\},\{(0,0),(1,0),(0,1),(1,1)\}\}
	\end{eqnarray*}
\end{solucao}

\begin{exercicio}
	content
\end{exercicio}
\begin{solucao}
	content
\end{solucao}

\begin{exercicio}
	Prove que $x^0=1$, para todo conjunto $x$, e explique o que isso significa.
\end{exercicio}
\begin{solucao}
	content
\end{solucao}

\begin{exercicio}
	content
\end{exercicio}
\begin{solucao}
	content
\end{solucao}

\begin{exercicio}
	Para qual(is) das seguintes definições alternativas de pares ordenados o teorema 4.2 continua valendo? Justifique.
	\begin{enumerate}[label=(\alph{*})]
		\item $(a,b)=\{a,\{b\}\}$;
		\item $(a,b)=\{\{a\},\{b\}\}$;
		\item $(a,b)=\{\{a\},\{\{b\}\}\}$;
		\item $(a,b)=\{\{0,a\},\{1,b\}\}$.
	\end{enumerate}
\end{exercicio}
\begin{solucao}
	content
\end{solucao}

\begin{exercicio}
	content
\end{exercicio}
\begin{solucao}
	content
\end{solucao}

\begin{exercicio}
	Dê exemplo de uma função injetora de $\omega$ em $\omega$ que não é injetora.
\end{exercicio}
\begin{solucao}
	Pelo Teorema Fundamental da Aritmética, todo número natural $n\geq2$ pode ser escrito como um produto de números primos $n=p_1\cdots p_m$, e este produto é único a menos da ordem dos fatores. Esse teorema nos permite definir uma função $f:\omega\to\omega$ pondo $f(0)=0$, $f(1)=1$ e, para cada $n\geq2$, $f(n)=m$, em que $m$ é a quantidade de fatores primos na decomposição de $n$. $f$ é sobrejetiva, pois, por exemplo, para cada $m\in\omega$, $f(2^m)=m$, mas $f$ não é injetiva, pois, por exemplo, $f(2^m)=f(3^m)$ e $2^m\neq 3^m$ se $m\geq1$.
\end{solucao}

\begin{exercicio}
	content
\end{exercicio}
\begin{solucao}
	content
\end{solucao}

\begin{exercicio}
	Seja $X$ um conjunto e sejam $x_0$ e $y_0$ dois elementos distintos de $X$. Considere a relação em $X$:
	$$R=\{(x,y)\in X\times X:x=y\}\cup\{(x_0,y_0),(y_0,x_0)\}$$
	\begin{enumerate}[label=(\alph{*})]
		\item Prove que $R$ é uma relação de equivalência em $X$.
		\item Descreva os elementos de $X/R$.
	\end{enumerate}
\end{exercicio}
\begin{solucao}
	\begin{enumerate}[label=(\alph{*})]
		\item Devemos demonstrar que a relação $R$ é reflexiva, simétrica e transitiva. Em linguagem de primeira ordem:
		$$xRy\leftrightarrow(x=y)\vee(x=x_0\wedge y=y_0)\vee(x=y_0\wedge y=x_0).$$
		
		\textbf{Reflexividade:} Para qualquer $x\in X$, $x=x$, logo $xRx$.
		
		\textbf{Simetria:} Se $xRy$, temos os seguintes casos:
		\begin{enumerate}[label=Caso \arabic{*}.]
			\item $x=y$: Então $y=x$, logo $yRx$;
			\item $x=x_0$: Então $y=x_0$ ou $y=y_0$. Se $y=x_0$, teremos $y=x$, logo $yRx$. Se $y=y_0$, o par $(x_0,y_0)\in R$ pela definição de $R$. Como $(y_0,x_0)\in R$ por definição, temos que vale $y_0Rx_0$, logo $yRx$.
			\item $x=y_0$: É análogo ao caso 2.
		\end{enumerate}
		
		\textbf{Transitividade:} Se $xRy$ e $yRz$, temos os casos
		\begin{enumerate}[label=Caso \arabic{*}.]
			\item $x\neq x_0$ e $y\neq y_0$: Então $x=y$ e $y=z$, logo $x=z$ e, portanto, $xRz$.
			\item $x=x_0$ e $y\neq y_0$: Se $x=x_0$, então $y=x_0$ ou $y=y_0$, mas pela hipótese deste caso, deve-se ter $y=x_0$. Como também $yRz$, então $y=z$ ou $y=y_0$ e $z=x_0$ (este caso está descartado), ou $y=x_0$ e $z=y_0$. Então $z=x_0$ (logo $zRx$ que, por reflexividade, dá $xRz$) ou $z=y_0$ (que, pela definição de $R$, $x_0Ry_0$, logo $xRz$). Em qualquer caso $xRz$, que desejávamos.
			\item $x\neq x_0$ e $y=y_0$: Análogo ao caso 2.
			\item $x=x_0$ e $y=y_0$: Como $yRz$, então $z=y_0$ ou $z=x_0$. Em qualquer caso $xRz$ pela definição de $R$.
		\end{enumerate}
		
		Isto conclui a prova de que $R$ é uma relação de equivalência.
		
		\item Lembre que $[x]_R=\{y\in X:xRy\}$. Dividimos novamente em casos.
		\begin{enumerate}[label=Caso \arabic{*}.]
			\item $x\neq x_0$ e $x\neq y_0$: Então $xRy\leftrightarrow x=y$, logo $[x]_R=\{x\}$.
			\item $x= x_0$ e $x\neq y_0$: Então $xRy\leftrightarrow (x=y)\vee (y=y_0)$, logo $[x]_R=\{x_0,y_0\}$.
			\item $x\neq x_0$ e $x= y_0$: Então $xRy\leftrightarrow (x=y)\vee (y=x_0)$, logo $[x]_R=\{x_0,y_0\}$.
		\end{enumerate}
		
		Em resumo, $X/R=\{\{x\}:x\in X\setminus\{x_0,y_0\}\}\cup\{\{x_0,y_0\}\}.$
	\end{enumerate}
\end{solucao}

\begin{exercicio}
	content
\end{exercicio}
\begin{solucao}
	content
\end{solucao}

\begin{exercicio}
	Como fica uma relação de equivalência sobre $\emptyset$? Ela satisfaz o Teorema 4.28?
\end{exercicio}
\begin{solucao}
	content
\end{solucao}

\begin{exercicio}
	content
\end{exercicio}
\begin{solucao}
	content
\end{solucao}

\begin{exercicio}
	Dê exemplo ou prove que não existe.
	\begin{enumerate}[label=(\alph{*})]
		\item Uma ordem total que não é uma boa ordem.
		\item Uma árvore que não é uma ordem total.
		\item Um reticulado que não é árvore.
		\item Uma árvore que é um reticulado mas não é totalmente ordenado.
	\end{enumerate}
\end{exercicio}
\begin{solucao}
	content
\end{solucao}

\begin{exercicio}
	content
\end{exercicio}
\begin{solucao}
	content
\end{solucao}

\begin{exercicio}
	Prove que $1+1=2$.
\end{exercicio}
\begin{solucao}
	\begin{eqnarray*}
		1+1&=&s(1)(1)=s(1)(0^+)=(s(1)(0))^+=1^+ \\
		&=&1\cup\{1\}=\{0\}\cup\{\{0\}\}=\{0,\{0\}\} \\
		&=&\{0,1\} \\
		&=&2.
	\end{eqnarray*}
\end{solucao}

\begin{exercicio}
	content
\end{exercicio}
\begin{solucao}
	content
\end{solucao}

\begin{exercicio}
	Use o teorema da recursão e as operações de adição e multiplicação para definir cada uma das seguintes funções de domínio $\omega$. Em cada item, utilize a versão do teorema da recursão mais adequada.
	\begin{enumerate}[label=(\alph{*})]
		\item a função $f(n)=2^n$;
		\item o fatorial, isto é, $f(n)=n!$, onde $0!=1$ e $(n+1)!=(n+1)\cdot n!$;
		\item a sequência de Fibonacci, ou seja, a função $f:\omega\to\omega$ tal que $f(0)=f(1)=1$ e $f(n+2)=f(n)+f(n+1)$.
	\end{enumerate}
\end{exercicio}
\begin{solucao}
	\begin{enumerate}[label=(\alph{*})]
		\item Para cada $n$ podemos escrever $f(n+1)=2^{n+1}=2\cdot 2^n=2\cdot f(n)$. Isto nos motiva a definir $g:\omega\to\omega$ por $g(x)=2x$. Pelo Teorema da Recursão Finita, existe uma única função $f:\omega\to\omega$ tal que $f(0)=1$ e $f(n+1)=g(f(n))=2\cdot f(n)$.
		\item Para cada $n$ podemos escrever $f(n+1)=(n+1)!=(n+1)\cdot n!=(n+1)\cdot f(n)$. Isto nos motiva a definir $g:\omega\times\omega\to\omega$ por $g(x,y)=(x+1)\cdot y$. Pelo Teorema da Recursão Finita com Parâmetro, existe uma única função $f:\omega\to\omega$ tal que $f(0)=1$ e $f(n+1)=g(n,f(n))=(n+1)\cdot f(n)$.
		\item Defina $g:\omega^{<\omega}\to \omega$ por
		$$
		g(x)=\left\{
			\begin{array}{ll}
				1,&\mathrm{se}\ \dom x=0 \vee \dom x=1 \\
				x(n-1)+x(n-2),& \mathrm{se}\ \dom x=n>1.
			\end{array}
		\right.
		$$
		Pelo Teorema da Recursão Completa, existe uma única função $f:\omega\to\omega$ tal que $f(n)=g(f|n)$. Perceba que
		$$f(0)=g(f|0)=1 \qquad \mathrm{e} \qquad f(1)=g(f|1)=1,$$
		pois, pela definição de $g$, $g(x)=1$ se $\dom x=0$ ou $\dom x=1$. Para $n>1$, pela definição de $g$,
		$f(n)=g(f|n)=f(n-1)+f(n-2)$. Isto mostra que a função descrita no exercício existe de fato.
	\end{enumerate}
\end{solucao}
	\include{capitulos/cap5}
	\include{capitulos/cap6}
	\include{capitulos/cap7}
	\include{capitulos/cap8}
	\include{capitulos/cap9}
	\include{capitulos/cap10}
	\include{capitulos/cap11}
	\include{capitulos/cap12}
	
	%\bibliographystyle{amsplain}
	%\printbibliography[title=Referências]
	%\bibliography{bibliografia.bib}
	\begin{thebibliography}{10}
		\bibitem[Fa]{Fajardo} FAJARDO, R. A. dos S. \textit{A Teoria dos Conjuntos e os Fundamentos da Matemática}. São Paulo: Edusp, 2024.
		\bibitem[Li]{Lima16} LIMA, E. L. \textit{Curso de análise}. 14 ed. Rio de Janeiro: Impa, 2016. v. 1.
	\end{thebibliography}
\end{document}